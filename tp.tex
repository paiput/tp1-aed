\documentclass[10pt,a4paper]{article}

\input{AEDmacros}
\usepackage{caratula}

\titulo{Trabajo Práctica 1}
\subtitulo{Especificación de TADs}

\fecha{\today}

\materia{Materia de la carrera}
\grupo{Grupo Almendra}

\integrante{Apellido, Nombre1}{001/01}{email1@dominio.com}
\integrante{Apellido, Nombre2}{002/01}{email2@dominio.com}
\integrante{Apellido, Nombre3}{003/01}{email3@dominio.com}
\integrante{Yu, Patricio}{1247/24}{yupatricio0@gmail.com}

% Document
\begin{document}
\maketitle

\section{Consideraciones}
\subsection{Renombre de tipo: \textit{transaccion}}
Por definicion del problema sabemos que una transaccion es una cuadru-pla que tiene la siguiente estructura:

\vspace{0.3cm}
\noindent
\fbox{
    \begin{minipage}{\dimexpr\textwidth-2\fboxsep-2\fboxrule} % Ajusta el ancho al texto
        transaccion: $<$id\_transaccion: \ent, id\_comprador: \ent, id\_vendedor: \ent, monto: \ent$>$
    \end{minipage}
}
\vspace{0.1cm}

Lo que hacemos es un \textbf{renombre de tipo} para referirnos a esta cuadru-pla como un tipo \textcolor{red}{\textit{transaccion}}.

\subsection{Renombre de tipo: \textit{bloque}}
Un bloque tiene un \textit{id\_bloque} que se representa con un entero y puede tener hasta 50 \textit{transacciones}:

\vspace{0.3cm}
\noindent
\fbox{
    \begin{minipage}{\dimexpr\textwidth-2\fboxsep-2\fboxrule} % Ajusta el ancho al texto
        bloque: $<$id\_bloque: \ent, transacciones: \TLista{transaccion}$>$
    \end{minipage}
}
\vspace{0.1cm}

Luego realizaremos un \textbf{renombre de tipo} para referirnos a esta du-pla como un tipo \textcolor{red}{\textit{bloque}}.

\vspace{0.3cm}

\noindent
\fbox{
    \begin{minipage}{\dimexpr\textwidth-2\fboxsep-2\fboxrule}
        TAD \$Berretacoin \{ \\
        \null \qquad obs bloques: \TLista{bloque} \\
        \null \qquad obs usuarios: \TLista{usuario}

        \setlength{\leftskip}{2em} \begin{proc}{agregarBloque}{}{\ent \space \{}
            \requiere{True}
            \asegura{True}
        \end{proc}\}

        \setlength{\leftskip}{0pt}\}
    \end{minipage}
}

\end{document}