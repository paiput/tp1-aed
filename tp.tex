\documentclass[10pt,a4paper]{article}

\usepackage[spanish,activeacute,es-tabla]{babel}
\usepackage[utf8]{inputenc}
\usepackage{ifthen}
\usepackage{listings}
\usepackage{dsfont}
\usepackage{subcaption}
\usepackage{amsmath}
\usepackage[strict]{changepage}
\usepackage[top=1cm,bottom=2cm,left=1cm,right=1cm]{geometry}%
\usepackage{color}%
\newcommand{\tocarEspacios}{%
	\addtolength{\leftskip}{3em}%
	\setlength{\parindent}{0em}%
}

% Especificacion de procs

\newcommand{\In}{\textsf{in }}
\newcommand{\Out}{\textsf{out }}
\newcommand{\Inout}{\textsf{inout }}

\newcommand{\encabezadoDeProc}[4]{%
	% Ponemos la palabrita problema en tt
	%  \noindent%
	{\normalfont\bfseries\ttfamily proc}%
	% Ponemos el nombre del problema
	\ %
	{\normalfont\ttfamily #2}%
	\
	% Ponemos los parametros
	(#3)%
	\ifthenelse{\equal{#4}{}}{}{%
		% Por ultimo, va el tipo del resultado
		\ : #4}
}

\newenvironment{proc}[4][res]{%
	
	% El parametro 1 (opcional) es el nombre del resultado
	% El parametro 2 es el nombre del problema
	% El parametro 3 son los parametros
	% El parametro 4 es el tipo del resultado
	% Preambulo del ambiente problema
	% Tenemos que definir los comandos requiere, asegura, modifica y aux
	\newcommand{\requiere}[2][]{%
		{\normalfont\bfseries\ttfamily requiere}%
		\ifthenelse{\equal{##1}{}}{}{\ {\normalfont\ttfamily ##1} :}\ %
		\{\ensuremath{##2}\}%
		{\normalfont\bfseries\,\par}%
	}
	\newcommand{\asegura}[2][]{%
		{\normalfont\bfseries\ttfamily asegura}%
		\ifthenelse{\equal{##1}{}}{}{\ {\normalfont\ttfamily ##1} :}\
		\{\ensuremath{##2}\}%
		{\normalfont\bfseries\,\par}%
	}
	\renewcommand{\aux}[4]{%
		{\normalfont\bfseries\ttfamily aux\ }%
		{\normalfont\ttfamily ##1}%
		\ifthenelse{\equal{##2}{}}{}{\ (##2)}\ : ##3\, = \ensuremath{##4}%
		{\normalfont\bfseries\,;\par}%
	}
	\renewcommand{\pred}[3]{%
		{\normalfont\bfseries\ttfamily pred }%
		{\normalfont\ttfamily ##1}%
		\ifthenelse{\equal{##2}{}}{}{\ (##2) }%
		\{%
		\begin{adjustwidth}{+5em}{}
			\ensuremath{##3}
		\end{adjustwidth}
		\}%
		{\normalfont\bfseries\,\par}%
	}
	
	\newcommand{\res}{#1}
	\vspace{1ex}
	\noindent
	\encabezadoDeProc{#1}{#2}{#3}{#4}
	% Abrimos la llave
	\par%
	\tocarEspacios
}
{
	% Cerramos la llave
	\vspace{1ex}
}

\newcommand{\aux}[4]{%
	{\normalfont\bfseries\ttfamily\noindent aux\ }%
	{\normalfont\ttfamily #1}%
	\ifthenelse{\equal{#2}{}}{}{\ (#2)}\ : #3\, = \ensuremath{#4}%
	{\normalfont\bfseries\,;\par}%
}

\newcommand{\pred}[3]{%
	{\normalfont\bfseries\ttfamily\noindent pred }%
	{\normalfont\ttfamily #1}%
	\ifthenelse{\equal{#2}{}}{}{\ (#2) }%
	\{%
	\begin{adjustwidth}{+2em}{}
		\ensuremath{#3}
	\end{adjustwidth}
	\}%
	{\normalfont\bfseries\,\par}%
}

% Tipos

\newcommand{\nat}{\ensuremath{\mathds{N}}}
\newcommand{\ent}{\ensuremath{\mathds{Z}}}
\newcommand{\float}{\ensuremath{\mathds{R}}}
\newcommand{\bool}{\ensuremath{\mathsf{Bool}}}
\newcommand{\cha}{\ensuremath{\mathsf{Char}}}
\newcommand{\str}{\ensuremath{\mathsf{String}}}

% Logica

\newcommand{\True}{\ensuremath{\mathrm{true}}}
\newcommand{\False}{\ensuremath{\mathrm{false}}}
\newcommand{\Then}{\ensuremath{\rightarrow}}
\newcommand{\Iff}{\ensuremath{\leftrightarrow}}
\newcommand{\implica}{\ensuremath{\longrightarrow}}
\newcommand{\IfThenElse}[3]{\ensuremath{\mathsf{if}\ #1\ \mathsf{then}\ #2\ \mathsf{else}\ #3\ \mathsf{fi}}}
\newcommand{\yLuego}{\land_L}
\newcommand{\oLuego}{\lor _L}
\newcommand{\implicaLuego}{\implica_L}

\newcommand{\cuantificador}[5]{%
	\ensuremath{(#2 #3: #4)\ (%
		\ifthenelse{\equal{#1}{unalinea}}{
			#5
		}{
			$ % exiting math mode
			\begin{adjustwidth}{+2em}{}
				$#5$%
			\end{adjustwidth}%
			$ % entering math mode
		}
		)}
}

\newcommand{\existe}[4][]{%
	\cuantificador{#1}{\exists}{#2}{#3}{#4}
}
\newcommand{\paraTodo}[4][]{%
	\cuantificador{#1}{\forall}{#2}{#3}{#4}
}

%listas

\newcommand{\TLista}[1]{\ensuremath{seq \langle #1\rangle}}
\newcommand{\lvacia}{\ensuremath{[\ ]}}
\newcommand{\lv}{\ensuremath{[\ ]}}
\newcommand{\longitud}[1]{\ensuremath{|#1|}}
\newcommand{\cons}[1]{\ensuremath{\mathsf{addFirst}}(#1)}
\newcommand{\indice}[1]{\ensuremath{\mathsf{indice}}(#1)}
\newcommand{\conc}[1]{\ensuremath{\mathsf{concat}}(#1)}
\newcommand{\cab}[1]{\ensuremath{\mathsf{head}}(#1)}
\newcommand{\cola}[1]{\ensuremath{\mathsf{tail}}(#1)}
\newcommand{\sub}[1]{\ensuremath{\mathsf{subseq}}(#1)}
\newcommand{\en}[1]{\ensuremath{\mathsf{en}}(#1)}
\newcommand{\cuenta}[2]{\mathsf{cuenta}\ensuremath{(#1, #2)}}
\newcommand{\suma}[1]{\mathsf{suma}(#1)}
\newcommand{\twodots}{\ensuremath{\mathrm{..}}}
\newcommand{\masmas}{\ensuremath{++}}
\newcommand{\matriz}[1]{\TLista{\TLista{#1}}}
\newcommand{\seqchar}{\TLista{\cha}}

\renewcommand{\lstlistingname}{Código}
\lstset{% general command to set parameter(s)
	language=Java,
	morekeywords={endif, endwhile, skip},
	basewidth={0.47em,0.40em},
	columns=fixed, fontadjust, resetmargins, xrightmargin=5pt, xleftmargin=15pt,
	flexiblecolumns=false, tabsize=4, breaklines, breakatwhitespace=false, extendedchars=true,
	numbers=left, numberstyle=\tiny, stepnumber=1, numbersep=9pt,
	frame=l, framesep=3pt,
	captionpos=b,
}

\usepackage{caratula}
\usepackage{hyperref}

\titulo{Trabajo Práctica 1}
\subtitulo{Especificación de TADs}

\fecha{\today}

\materia{Materia de la carrera}
\grupo{Grupo Almendra}

\integrante{Apellido, Nombre1}{001/01}{email1@dominio.com}
\integrante{Apellido, Nombre2}{002/01}{email2@dominio.com}
\integrante{Apellido, Nombre3}{003/01}{email3@dominio.com}
\integrante{Yu, Patricio}{1247/24}{yupatricio0@gmail.com}

% Document
\begin{document}
\maketitle

\section{Renombres de tipo}
En lugar de \textcolor{red}{\textit{tuplas}} utilizaremos \textcolor{red}{\textit{strutcs}} para una mejor legibilidad de la especificacion a lo largo del documento.
\subsection{Tipo: \textit{usuario}}
El usuario consta de un \textit{id\_usuario} y una cantidad de \textit{cant\_berretacoin}, ambos enteros positivos, realizaremos el siguiente renombre de tipo:

\vspace{0.3cm}
\noindent
\fbox{
    \begin{minipage}{\dimexpr\textwidth-2\fboxsep-2\fboxrule}
        \textit{usuario} \textbf{ES} struct$<$id\_usuario: \ent, cant\_berretacoin: \ent$>$
    \end{minipage}
}
\vspace{0.1cm}

\subsection{Tipo: \textit{transaccion}}
Por definicion del problema sabemos que una transaccion es una cuadru-pla que consta de un \textit{id\_transaccion}, \textit{id\_comprador}, \textit{id\_vendedor}, \textit{monto}, todos enteros positivos, realizaremos el siguiente renombre de tipo:

\vspace{0.3cm}
\noindent
\fbox{
    \begin{minipage}{\dimexpr\textwidth-2\fboxsep-2\fboxrule}
        \textit{transaccion} \textbf{ES} struct$<$id\_transaccion: \ent, id\_comprador: \ent, id\_vendedor: \ent, monto: \ent$>$
    \end{minipage}
}
\vspace{0.1cm}

\subsection{Tipo: \textit{bloque}}
Un bloque tiene un \textit{id\_bloque} que se representa con un entero positivo y puede tener hasta 50 \textit{transacciones}:

\vspace{0.3cm}
\noindent
\fbox{
    \begin{minipage}{\dimexpr\textwidth-2\fboxsep-2\fboxrule}
        \textit{bloque} \textbf{ES} struct$<$id\_bloque: \ent, transacciones: \TLista{transaccion}$>$
    \end{minipage}
}
\vspace{0.1cm}

\section{Definicion del TAD}
\noindent
\fbox{
    \begin{minipage}{\dimexpr\textwidth-2\fboxsep-2\fboxrule}
        \raggedright
        \vspace{0.2cm}
        TAD Berretacoin \{ \\
        \qquad obs bloques: \TLista{bloque} \\
        \qquad obs usuarios: \TLista{usuario} \\
        % ----------------PROCEDIMIENTOS----------------
        \setlength{\leftskip}{2em}
        \begin{proc}{nuevoBerretacoin}{\In{usuarios: \TLista{usuario}}}{Berretacoin}
            \asegura{res.bloques = \big[ \ \big]}
            \asegura{res.usuarios = usuarios \ \land \ |usuarios| \ge 1}
            \asegura{\paraTodo[unalinea]{i}{\ent}{(0 \le i < |res.usuarios|) \implicaLuego (res.usuarios[i].cant\_berretacoin = 0)}}
            \asegura{\paraTodo[unalinea]{i}{\ent}{(0 \le i < |res.usuarios|) \implicaLuego (esEnteroPositivo(res.usuarios[i].id\_usuario))\yLuego \lnot (hayRepetidos(res.usuarios,\ res.usuarios[i].id\_usuario))}}
        \end{proc} \}
        \vspace{0.3em}
        \begin{proc}{agregarUsuario}{\Inout{berretacoin: Berretacoin}, \In{u: usuario}}{}

        \end{proc} \}
        \vspace{0.3em}
        \begin{proc}{agregarBloque}{}{\ent}
            \requiere{True}
            \asegura{True}
        \end{proc} \} \par
        \vspace{0.5em}
        % ----------------PREDICADOS----------------
        \pred{hayRepetidos}{s: \TLista{T}, e: T}{
            \qquad\quad \left(\sum_{i = 0}^{|s| - 1} ifThenElseFi(s[i] = e, 1, 0)\right) \ge 2
        }
        \vspace{0.5em}
        \pred{esEnteroPositivo}{n: \ent}{
            \qquad\quad x \ge 0
        }
        \setlength{\leftskip}{0pt}\}
        \vspace{0.2cm}
    \end{minipage}
}

\end{document}